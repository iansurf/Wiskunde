
\documentclass{article}
\usepackage{amsmath}

\begin{document}

\section*{Verification: Rotation Matrix Forms a Group}

The rotation matrix is given by:
\begin{equation}
R(a) = \begin{bmatrix}
\cos(a) & -\sin(a) \\
\sin(a) & \cos(a)
\end{bmatrix}.
\end{equation}
The set of all such matrices corresponds to rotations by any angle \(a \in \mathbb{R}\). We verify the group properties under matrix multiplication:

\subsection*{1. Closure}
If \( R(a) \) and \( R(b) \) are two rotation matrices, their product is:
\begin{align*}
R(a) \cdot R(b) &= \begin{bmatrix}
\cos(a) & -\sin(a) \\
\sin(a) & \cos(a)
\end{bmatrix}
\cdot
\begin{bmatrix}
\cos(b) & -\sin(b) \\
\sin(b) & \cos(b)
\end{bmatrix} \\
&= \begin{bmatrix}
\cos(a+b) & -\sin(a+b) \\
\sin(a+b) & \cos(a+b)
\end{bmatrix}.
\end{align*}
This is another rotation matrix \( R(a+b) \). Therefore, the set is closed under multiplication.

\subsection*{2. Associativity}
Matrix multiplication is associative. Hence, for all \( R(a), R(b), R(c) \),
\begin{equation}
R(a) \cdot (R(b) \cdot R(c)) = (R(a) \cdot R(b)) \cdot R(c).
\end{equation}

\subsection*{3. Identity Element}
The identity element is the rotation matrix corresponding to \(a = 0\):
\begin{equation}
R(0) = \begin{bmatrix}
1 & 0 \\
0 & 1
\end{bmatrix}.
\end{equation}
Multiplying \( R(0) \) with any \( R(a) \) leaves \( R(a) \) unchanged:
\begin{equation}
R(a) \cdot R(0) = R(0) \cdot R(a) = R(a).
\end{equation}

\subsection*{4. Inverse Element}
The inverse of \( R(a) \) is \( R(-a) \), the rotation by the negative angle:
\begin{equation}
R(-a) = \begin{bmatrix}
\cos(-a) & -\sin(-a) \\
\sin(-a) & \cos(-a)
\end{bmatrix} = \begin{bmatrix}
\cos(a) & \sin(a) \\
-\sin(a) & \cos(a)
\end{bmatrix}.
\end{equation}
Multiplying \( R(a) \cdot R(-a) \) or \( R(-a) \cdot R(a) \) results in the identity matrix.

\subsection*{Conclusion}
The set of all \( 2 \times 2 \) rotation matrices forms a \textbf{group} under matrix multiplication. This group is called the \textbf{special orthogonal group in two dimensions}, denoted as \( SO(2) \). It represents all possible rotations in \( \mathbb{R}^2 \) and satisfies all group properties.

\end{document}
